% !TEX root = dissertation_BB.tex
% \cleardoublepage
\chapter{Appendix B}
\markboth{\MakeUppercase{Appendix B}}{}
% \addcontentsline{toc}{chapter}{Appendix B}
\label{app:tables}
\setcounter{table}{0}
\renewcommand{\thetable}{B\arabic{table}}

\section*{Data sizes in microscopy}

\begin{table}[tbp]
  \begin{small}
    \renewcommand{\arraystretch}{2}
    \centering
    \begin{tabular}{rp{5cm}cccc}
        & \textbf{imaging device} & \textbf{image size} &  \parbox[c]{1.2cm}{\textbf{frame}\\ \textbf{rate}} & \textbf{data rate} & \parbox[c]{1.2cm}{\textbf{data\\ size}} \\
        \hline
        \hline
        \textbf{SPIM} & 2x sCMOS camera (e.g. Hamamatsu ORCA Flash4.0) & 2048x2048 & 50/s & 800 MB/s & 10 TB \\ \hline
        \textbf{SMLM} & 2x EMCCD camera (e.g. Andor iXon Ultra 897) & 512x512 & 56/s & 56 MB/s & 500 GB \\ \hline
        \textbf{screening} & CCD camera (e.g. Hamamatsu ORCA-R2) & 1344x1024 & 8.5s/ & 22 MB/s & 5 TB \\ \hline
        \textbf{confocal} & Zeiss LSM 880, 10 channels & 512x512 & 5/s & 12.5 MB/s & 50 GB \\ 
    \end{tabular}
    \bcaption[Data sizes in microscopy]{Typical devices used for confocal microscopy, high-content screening, single-molecule localization microscopy and light-sheet microscopy and their data production characteristics. Data visualized on Figure \ref{fig:sizes}}
    \label{tab:sizes}
  \end{small}
\end{table}


  
\section*{Lossless compression performance}

\begin{table}[tbp]
  \renewcommand{\arraystretch}{2}
  \setlength{\tabcolsep}{9pt}
  \centering
  \begin{tabular}{lrrrr}
      & \textbf{write speed} & \textbf{read speed} & \textbf{CR} & \textbf{file size} \\
      \hline
      \hline
      \textbf{\b3d} & 1,115.08 MB/s & 928.97 MB/s & 9.861 & 100\% \\ \hline
      \textbf{KLB} & 283.19 MB/s & 619.95 MB/s & 10.571 & 93.28\% \\ \hline
      \textbf{JPEG2000} & 31.94 MB/s & 26.38 MB/s & 11.782 & 83.69\% \\ \hline
      \textbf{TIFF uncompressed} & 202.32 MB/s & 161.08 MB/s & 1.00 & 986.1\% \\ \hline
      \textbf{TIFF + LZW} & 40.85 MB/s & 102.37 MB/s & 5.822 & 169.37\%
  \end{tabular}
  \bcaption[Lossless compression performance]{\b3d is compared with various popular lossless image compression methods regarding write speed, read speed and compression ratio (original size / compressed size). Data visualized on Figure \ref{fig:bubbles}.}
  \label{tab:performance}
\end{table}


\section*{Benchmarking datasets}

\begin{table}[tbp]
  \begin{small}
    \renewcommand{\arraystretch}{2}
    \centering
    \begin{tabular}{llp{7cm}r}
      \textbf{Dataset name} & \parbox[c]{2cm}{\textbf{Imaging\\modality}} & \textbf{Description} & \textbf{Size (MB)} \\
      \hline
      \hline
      \textbf{drosophila} & SPIM & dataset acquired in MuVi-SPIM of a Drosophila melanogaster embryo expressing H2Av-mCherry nuclear marker & 494.53 \\ \hline
      \textbf{zebrafish} & SPIM & dataset acquired in MuVi-SPIM of a zebrafish embryo expressing b-actin::GCaMP6f calcium sensor & 2,408.00 \\ \hline
      \textbf{phallusia} & SPIM & dataset acquired in MuVi-SPIM of a Phallusia mammillata embryo expressing PH-citrine membrane marker & 1,323.88  \\ \hline
      \textbf{simulation} & SMLM & MT0.N1.LD-2D simulated dataset of microtubules labeled with Alexa Fluor 647 from SMLMS 2016 challenge & 156.22 \\ \hline
      \textbf{microtubules} & SMLM & microtubules immuno-labeled with Alexa Fluor 674-bound antibodies in U2OS cells & 1,643.86  \\ \hline
      \textbf{lifeact} & SMLM & actin network labeled with LifeAct-tdEOS in U2OS cells & 3,316.15  \\ \hline
      \textbf{dapi} & screening & wide field fluorescence images of DAPI stained HeLa Kyoto cells \cite{simpson_genome-wide_2012} & 1,005.38 \\ \hline
      \textbf{vsvg} & screening & wide field fluorescence images of CFP-tsO45G proteins in HeLa Kyoto cells \cite{simpson_genome-wide_2012} & 1,005.38  \\ \hline
      \textbf{membrane} & screening & wide field fluorescence images of membrane localized CFP-tsO45G proteins labeled with AlexaFluor647 in HeLa Kyoto cells \cite{simpson_genome-wide_2012} & 1,005.38  \\ 
    \end{tabular}
    \bcaption[Datasets used for benchmarking compression performance]{}
    \label{tab:datasets}
  \end{small}
\end{table}