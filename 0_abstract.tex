% !TEX root = dissertation_BB.tex
%% spellcheck-language en-US

\chapter*{Abstract}
\addcontentsline{toc}{chapter}{Abstract}
Light-sheet fluorescence microscopy, also called single plane illumination microscopy, has numerously proven its usefulness for long term imaging of embryonic development. This thesis tackles two challenges of light-sheet microscopy: high resolution isotropic imaging of delicate, light-sensitive samples, and real-time image processing and compression of light-sheet microscopy images.

A symmetric light-sheet microscope is presented, featuring two high numerical aperture objectives arranged in \SI{120}{\degree}, both capable of illuminating the sample and detecting the fluorescence signal. It allows for multi-view imaging of delicate samples where rotation is not possible. The optical properties of the microscope are characterized, and its imaging capabilities are demonstrated on \textit{Drosophila melanogaster} embryos and mouse zygotes.

To address the big data problem of light-sheet microscopy, a real-time, GPU-based image processing pipeline is presented. It is capable of performing commonly required preprocessing tasks such as fusion of opposing views during image acquisition. It includes a novel, high-speed image compression method that alongside a lossless mode also has a noise dependent lossy mode. This allows to significantly increase the compression ratio without affecting the results of any further analysis. A detailed performance analysis is presented of the different compression modes for various biological samples and imaging modalities.



% preprocessing and compressing the data real-time
%   parallel processing, GPU
%   reduce time for evaluation
%   reduce necessary storage and costs
%   noise dependent lossy compression
%   within noise level compression - although not lossless in the mathematical sense, but the introduced differences for each pixel are within the range of uncertainty of the measurement

% % "Fluorescence imaging techniques such as single molecule localization microscopy, high-content screening and light-sheet microscopy are producing ever-larger datasets, which poses increasing challenges in data handling and data sharing. Here, we introduce
% A real-time compression library is introduced that allows for very fast (beyond 1 GB/s) compression and decompression of microscopy datasets during acquisition. In addition to an efficient lossless mode, the algorithm also includes a lossy option, which limits pixel deviations to the intrinsic noise level of the image and yields compression ratio of up to 100-fold. A detailed performance analysis is presented of the different compression modes for various biological samples and imaging modalities.




% spellchecker:disable
\chapter*{Tartlami kivonat}
\addcontentsline{toc}{chapter}{Tartlami kivonat}
Absztrakt magyarul.
% spellchecker: enable