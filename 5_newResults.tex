% !TEX root = dissertation_BB.tex
%% spellcheck-language en-US

\section{New scientific results}

  \paragraph{Thesis I.}\textit{I have designed and constructed a new light-sheet microscope suitable for high resolution imaging of delicate samples. A novel arrangement of two high numerical aperture objectives in 120 degrees combined with a tilted light-sheet allows for near isotropic resolution while increasing light collection efficiency by a factor of two.}
  
    Corresponding publications: \cite{hoyer_breaking_2016}, \cite{strnad_inverted_2016}, \cite{de_medeiros_light-sheet_2016} 

    Dual Mouse-SPIM is a novel design for symmetric light-sheet microscopy. The use of high NA objectives in \SI{120}{\degree} not only increases volumetric resolution compared to the conventional \SI{90}{\degree} setup, but due to the larger detection angle, light collection efficiency is doubled. This is especially beneficial for delicate, light-sensitive specimens, such as mouse embryos, since phototoxic effects are reduced while the contrast is preserved.

    I designed the optical path, the layout, and the custom mechanical components, and constructed the microscope. As part of the microscope I have designed a custom beam splitter unit that allows the use of a single galvanometric scanner to generate light-sheets for both objectives. I have also designed a custom detection merging unit that allows the use of a single camera for both detection views.

    I have characterized the optical properties of the microscope, measured the illumination profile and point spread function. With a \SI{3.6}{\micro m} thick light-sheet, a \SI{95}{\micro m} field of view is evenly illuminated. Dual view imaging of bead samples revealed a lateral resolution of \SI{314}{nm}, and axial resolution of \SI{496}{nm}. This is a 2.67\texttimes\ improvement compared to the axial resolution of a single detection objective lens. I have also demonstrated the imaging capabilities of the microscope on \textit{Drosophila melanogaster} embryos and mouse zygotes.


  \paragraph{Thesis II.} \textit{I have developed a GPU-based image processing pipeline for multi-view light-sheet microscopy that enables real-time fusion of opposing views.}

    Corresponding publications: \cite{balazs_gpu-based_2016}, \cite{balazs_gpu-based_2016-1}, \cite{balazs_gpu-based_2017}

    I have developed a GPU-based image preprocessing pipeline, which integrates directly to our universal microscope control software in LabVIEW. The pipeline currently supports background subtraction and background masking, furthermore it is capable of fusing opposing views of the same plane faster than real-time. I have shown that it is possible to reduce the registration of opposing camera views from a 3D alignment to a 2D alignment without any negative effects in image quality and resolution. This massively reduces the necessary computing resources, and allows the use of CUDA textures for faster than real-time image fusion. Processing speed of this implementation is \SI{138}{fps}, a 18.3\texttimes\ increase compared to a single threaded CPU implementation.



  \paragraph{Thesis Group III.} \textit{Real-time image compression.}
  \subparagraph{Thesis III/1.}
  \textit{I have  developed a new image compression algorithm that enables noise dependent lossy compression of light microscopy images, and can reach a compression ratio of 100 fold while preserving the results of downstream data analysis steps. A fast CUDA implementation allows for real-time image compression of high-speed microscopy images.}

    Corresponding publications: \cite{balazs_real-time_2017}, \cite{balazs_gpu-based_2016}, \cite{balazs_gpu-based_2016-1}, \cite{balazs_gpu-based_2017}
    
    % \b3d is an efficient, GPU-based image compression library allowing lossless and noise dependent lossy compression of microscopy images.
    Since many high-speed microscopy methods generate immense amounts of data, easily reaching terabytes per experiment, image compression is especially important to efficiently deal with such datasets. Existing compression methods suitable for microscopy images are not able to deal with the high data rate of modern sCMOS cameras ($\sim \SI{800}{MB/s}$).

    I developed a GPU-based parallel image compression algorithm called \b3d, capable of over \SI{1}{GB/s} throughput, allowing live image compression. To further reduce the data size, I developed a noise dependent lossy compression that only modifies the data in a deterministic manner. The allowed differences for each pixel can be specified as a proportion of the inherent image noise, accounting for photon shot noise and camera readout noise. Due to the use of pixel prediction, the subjective image quality is higher than for other methods that simply quantize the square root of the images.


  \subparagraph{Thesis III/2.} \textit{I have shown that within noise level compression does not significantly affect the results of most commonly used image processing tasks, and it allows a 3.32\texttimes\ average increase in compression ratio compared to lossless mode.}
  
    Corresponding publications: \cite{balazs_real-time_2017}, \cite{balazs_gpu-based_2016}, \cite{balazs_gpu-based_2016-1}, \cite{balazs_gpu-based_2017}
    
    As data integrity in microscopy is paramount for drawing the right conclusions from the experiments, using a lossy compression algorithm might be controversial.
    I have shown that the within noise level (WNL) mode of \b3d does not significantly affect the results of several commonly used image analysis tasks. For light-sheet microscopy data I have shown that WNL compression introduces less variation to the image than the photon shot noise. When segmenting nuclei of \textit{Drosophila} embryos and membranes of \textit{Phallusia} embryos, the overlap of the segmented regions of uncompressed and WNL compressed datasets were 99.6\% and 94.5\% respectively, while compression ratios were 19.83 for \textit{Drosophila} and 40.01 for \textit{Phallusia} embryos. For single molecule localization microscopy data I have shown that WNL compression only introduces 4\% increase in localization uncertainty, while the average compression ratio is increased from 1.44 (lossless) to 4.96 (WNL). I have also shown that change in localization error due to the compression does not depend on the SNR of the input images.
    


\section{Application of the results}
  Both the new Dual Mouse-SPIM microscope and the GPU-based image processing and compression pipeline have direct applications in light-sheet imaging of embryonic development.

  Multiple potential collaborators indicated their interest in using the Dual Mouse-SPIM for their studies in mouse embryonic development. In the context of the research of symmetry breaking events in the pre-implantation and early post-implantation stages, this system can be used for imaging larger specimens from multiple direction, which is not possible on previous microscopes, and could allow to observe previously unknown mechanisms.
  Another possible application is investigating chromosome missesgregation mechanisms in the first few divisions during embryonic development. The increased axial resolution of this system will allow to track each individual chromosome during the division process, which is not possible on current microscopes due to the insufficient axial resolution.

  The GPU-based image processing pipeline, especially the 2D fusion of opposing views is already being used on our lab's workhorse microscope, the MuVi-SPIM. Being able to fuse the two views of the opposing objectives during imaging not only results in considerable storage space savings, but significantly speeds up the data analysis as well.

  The image compression algorithm, \b3d, although was developed with light-sheet microscopy in mind, has a more wide-spread use-case. Any kind of high-speed, high-throughput light-microscopy experiment can benefit form the massive data reduction offered by the within-noise-level mode. Since the compression can also be done immediately during imaging, not only the storage requirements, but the data bandwidth is reduced as well, which renders the use of high performance RAID arrays and \SI{10}{Gbit} networks unnecessary, further reducing costs.
  Due to the similarly high decompression speed, reading the data is also accelerated, which can be beneficial for data browsing and 3D rendering applications. Several companies of different fields already expressed their interest in the compression library, including Bitplane AG (3D data analysis and visualization), Luxendo GmbH (light-sheet microscopy), and Hamamatsu Photonics K.K. (camera and sensor manufacturing).

% microscope:
% high res imaging of mouse development
% legfokepeppen kinetochore (chromosome) tracking
% Ellenberg group: kinetochore tracking, only works with isotropic res, not with orig. mouse-SPIM
% Hiiragi group: symmerty breaking, also for post-implantation, as the dual view allows to image larger specimens

% compression:
% any biology lab working with light-sheet microscopy
% time and space and money savings
% already in HDF5, works with Matlab, python, bigdataviewer
% commercial:
% Bitplane AG (Imaris, 3D data analysis software), Luxendo GmbH (light-sheet microscopes), Hamamatsu Photonics K.K (camera and sensor manufacturing)